%! Author = Len Washington III
%! Date = 2/07/2024
%! compiler = pdflatex

% Preamble
\documentclass[title={February 07, 2024 Notes}]{cs581notes}

% Document
\begin{document}

%<*2-07-2024>
\section{Evolutionary Algorithms [Wikipedia]}\label{sec:evolutionary-algorithms-[wikipedia]}
An evolutionary algorithm (EA) is a subset of evolutionary computation, a generic \emph{population-based metaheuristic optimization algorithm}.\\

An EA uses mechanisms inspired by biological evolution, such as \emph{reproduction, mutation, recombination, and selection.}
\emph{Candidate solutions to the optimization problem play the role of individuals} in a population,
and \emph{the fitness function determines the quality of the solutions} (see also loss function).\\

\emph{Evolution} of the population then takes place after the \emph{repeated application of the above operators}.

\chapter{Genetic Algorithm}\label{ch:genetic-algorithm}
\section{Components}\label{sec:genetic-algorithm-components}
\begin{table}[H]
    \centering
	\caption{Simple Genetic Algorithm}
	\label{tab:summary}
	\begin{tabular}{|>{\columncolor{black}}P{0.4\textwidth}|P{0.4\textwidth}|}
		\hline
		\textcolor{white}{\textbf{Representation}} & Bit strings\\
		\hline
		\textcolor{white}{\textbf{Recombination}} & 1-point crossover\\
		\hline
		\textcolor{white}{\textbf{Mutation}} & Bit flip\\
		\hline
		\textcolor{white}{\textbf{Parent Selection}} & Fitness proportional\\
		\hline
		\textcolor{white}{\textbf{Survival Selection}} & Generational\\
		\hline
	\end{tabular}
\end{table}

\subsection{Chromosome: Representation}\label{subsec:chromosome:-representation}
Individuals / chromosomes can be represented as a string of values.

\subsection{Binary Representation Issues}\label{subsec:binary-representation-issues}

\begin{table}[H]
	\centering
	\caption{Chromosome $A$}
	\label{tab:chromosome-a}
	\begin{tabular}{|*{10}{c|}}
		\hline
		\multicolumn{10}{|c|}{Variable/Gene = 512}\\
		\hline
		\textcolor{red}{1} & 0 & 0 & 0 & 0 & 0 & 0 & 0 & 0 & 0\\
		\hline
	\end{tabular}
\end{table}

\begin{table}[H]
	\centering
	\caption{Chromosome $A$}
	\label{tab:chromosome-a-mutation}
	\begin{tabular}{|*{10}{c|}}
		\hline
		\multicolumn{10}{|c|}{Variable/Gene = 0}\\
		\hline
		\textcolor{red}{0} & 0 & 0 & 0 & 0 & 0 & 0 & 0 & 0 & 0\\
		\hline
	\end{tabular}
\end{table}


Small, one bit, change can lead to drastic fitness changes.
Solution: \emph{use Gray Code}.

\subsection{Gray Code}\label{subsec:gray-code}
The reflected binary code (RBC), also known as reflected binary (RB) or Gray Code (named after Frank Gray -- Bell Labs),
is an \emph{ordering of the binary numeral system such that two successive values differ in only one bit} (binary digit).

\begin{table}[H]
	\centering
	\caption{Gray Code}
	\label{tab:gray-code}
	\begin{tabular}{|*{6}{P{0.15\textwidth}|}}
		\hline
		\rowcolor{black} \header{Decimal} & \header{Binary} & \header{Gray} & \header{Decimal} & \header{Binary} & \header{Gray}\\
		\hline
		0 & 0000 & 0000 & 7 & 0111 & 0100 \\
		\hline
		1 & 0001 & 0001 & 8 & 1000 & 1100 \\
		\hline
		2 & 0010 & 0011 & 9 & 1001 & 1101 \\
		\hline
		3 & 0011 & 0010 & 10 & 1010 & 1111 \\
		\hline
		4 & 0100 & 0110 & 11 & 1011 & 1110 \\
		\hline
		5 & 0101 & 0111 & 12 & 1100 & 1010 \\
		\hline
		6 & 0110 & 0101 & 13 & 1101 & 1011 \\
		\hline
	\end{tabular}
\end{table}


\subsubsection{Hamming Distance}\label{subsubsec:hamming-distance}
The \emph{Hamming distance} between \emph{two equal-length strings} of symbols is the \emph{number of positions at which the corresponding symbols are different}.\\

\emph{Gray code}: subsequent numbers $\rightarrow$ \emph{Hamming distance of 1}

\subsection{Integer Representation}\label{subsec:integer-representation}
\begin{itemize}
	\item Some problems naturally map (8-queens) to integer representations
	\begin{itemize}
		\item solution is an assignment variable = integer value
	\end{itemize}
	\item Unrestricted: any value is permissible
	\item Restricted: only values from a certain set / domain
	\begin{itemize}
		\item for example $\{0, 1, 2, 3\}$ for $\{\mbox{North, East, South, West}\}$
	\end{itemize}
	\item Consider:
	\begin{itemize}
		\item is there any relationship between values (e.g. 3 is more like 4 than 789 $\rightarrow$ ordinal relationship) or not (\{North, East, South, West\})
		\item Choose your recombination / mutation strategy accordingly
	\end{itemize}
\end{itemize}

\section{Permutation Encoding}\label{sec:permutation-encoding}
In permutation encoding, \emph{every chromosome is a string of numbers that represent a position in a sequence}.\\

Permutation encoding is useful for \emph{ordering problems}.
For some types of crossover and mutation corrections must be made to leave the chromosome consistent (i.e.\ have valid sequence in it) for certain problems.

\section{Genetic Algorithm: Other Selection Mechanisms}\label{sec:genetic-algorithm:-other-selection-mechanisms}
\subsection{Elitism}\label{subsec:elitism}
\emph{Elitism (Elitist selection)} is a strategy in genetic algorithms (in practice: evolutionary algorithms in general)
where one or more most fit individuals (the elites) in every generation, are inserted into the next generation \emph{without undergoing any change}.

This strategy usually speeds up the convergence of the algorithm.

\subsection{Steady State Selection}\label{subsec:steady-state-selection}
In every generation, few chromosomes are selected (good - with high fitness) for creating a new offspring.\\

Then some (bad - with low fitness) chromosomes are removed and the new offspring is placed in their place.\\

The rest of population survives to new generation.

\section{Fitness Proportional Selection: Issues}\label{sec:fitness-proportional-selection:-issues}
Fitness Proportional Selection approach has issues:
\begin{itemize}
	\item outstanding individuals take over the entire population quickly; this is called \emph{premature convergence}
	\item potential solutions
	\item when \emph{fitness values are all very close together}, there is almost \emph{no selection pressure} [= \emph{selection is almost uniformly random}] $\rightarrow$ performance increases slowly
	\item Potential solutions:
	\begin{itemize}
		\item windowing (slide and subtract some value based on history)
		\item Goldberg's sigma scaling: $f'(x) = \max\left( f(x) - (f_{avg} - 2\times\sigma_{f}), 0.0 \right)$
		\item \emph{ranking}
	\end{itemize}
\end{itemize}

\section{Rank Based Selection}\label{sec:rank-based-selection}
In rank selection, the \emph{selection probability does not depend directly on the fitness}, but \emph{on the fitness rank of an individual within the population}.
The exact fitness values themselves do not have to be available, but only a sorting of individuals according to quality.
\begin{itemize}
	\item Linear ranking:
	\[ P(R_{i}) = \frac{1}{n}\left( sp - (2sp - 2)\frac{i-1}{n-1} \right)\ \ 1\leq i\leq n,\ \ 1\leq sp\leq 2\ \ \mbox{with  } P(R_{i})\geq 0,\ \ \sum_{i=1}^{n} P(R_{i}) = 1 \]
	\begin{itemize}
		\item[]
		\definition{sp}{selection pressure (the degree to which the better individuals are favored: the higher the selectrion pressure, the more the better individuals are favored) which can take values between 1.0 (no selection pressure) and 2.0 (high selectrion pressure)}
	\end{itemize}
	\item Exponential ranking
\end{itemize}

\section{Replace worst}\label{sec:replace-worst}
\begin{itemize}
	\item In this scheme, the worst $k$ members of the population are selected for replacement
	\item This can lead to rapid improvements in the average population fitness, but can also cause premature convergence
	\begin{itemize}
		\item it will focus on most fit individual
	\end{itemize}
	\item Typically used with large populations
\end{itemize}

\section{Age-Based Replacement}\label{sec:age-based-replacement}
\begin{itemize}
	\item Rather than look at fitness of the individual, pick the oldest (in iterations) first to replace
	\item A FIFO queue will be needed
\end{itemize}

\section{Other Recombination Methods}\label{sec:other-recombination-methods}
\subsection{``Cut and Crossfill''}\label{subsec:``cut-and-crossfill''}
\begin{enumerate}[label=\arabic*.]
    \item Pick a random crossover point
	\item Cut both parents in two segments after this position
	\item Copy the first segment of Parent 1 into Child 1 and the first segment of Parent 2 into Child 2
	\item Scan Parent 2 from left to right and fill the second segment of Child 1 with values from Parent 2, skipping those that are already contained in it
	\item Do the same for Parent 1 and Child 2
\end{enumerate}

\subsection{Multiparent Recombination}\label{subsec:multiparent-recombination}
\begin{itemize}
	\item The idea: recombine genes of more than 2 parents
	\item Some strategies:
	\begin{itemize}
		\item Allele frequency-based: $\rho$-sexual voting
		\item segmentation and recombination of parents-based: the diagonal crossover
		\item Based on numerical operations on real-values alleles
	\end{itemize}
\end{itemize}

\section{Other Recombination Options}\label{sec:other-recombination-options}
\begin{itemize}
	\item Integer representation:
	\begin{itemize}
		\item same approaches as for binary
	\end{itemize}
	\item Floating-point representation:
	\begin{itemize}
		\item simple recombination
		\item simple arithmetic recombination
		\item whole arithmetic recombination
		\begin{itemize}
			\item Child 1: $\alpha\times x + (1-\alpha)\times y$ and Child 2: $\alpha\times y + (1-\alpha)\times x$
		\end{itemize}
	\end{itemize}
\end{itemize}

\subsection{Integer representation}\label{subsec:integer-representation-2}
Same approaches as for binary

\subsection{Floating-point representation:}\label{subsec:floating-point-representation:}
\begin{itemize}
	\item Simple recombination
	\item Simple arithmetic recombination
	\item Whole arithmetic recombination
	\[ \mbox{Child 1: } \alpha{}x \]
\end{itemize}

\section{Other Mutation Mechanisms}\label{sec:other-mutation-mechanisms}
\subsection{Scramble mutation}\label{subsec:scramble-mutation}
A subset of genes is chosen and their values randomly shuffles/scrambled.

\subsection{Inversion mutation}\label{subsec:inversion-mutation}
A subset of genes, a substring is selected and inverted.

\subsection{Interchage (order changing) mutation}\label{subsec:interchage-(order-changing)-mutation}
Randomly select two positions of the chromosome and interchange the values.

\section{Selected Problems}\label{sec:selected-problems}
\begin{center} \LARGE 8-Queens Problem \end{center}

\section{Artificial Neuron (Perceptron)}\label{sec:artificial-neuron-(perceptron)}
A (single-layer) perceptron is a model of a biological neuron.
It is made of the following components:
\begin{itemize}
	\item inputs $x_{i}$ - numerical values representing information
	\item weights $w_{i}$
\end{itemize}

\chapter{Genetic Programming}\label{ch:genetic-programming}
\section{Traditional Programming vs $\dots$}\label{sec:traditional-programming-vs}
Traditional Programming:
\begin{center} Input Data + Program (Rules) = Output \end{center}

$\dots$:
\begin{center} Input Data + Output = Program (Rules) \end{center}

%</2-07-2024>

\end{document}