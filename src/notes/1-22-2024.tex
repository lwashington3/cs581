%! Author = Len Washington III
%! Date = 1/22/2024
%! compiler = pdflatex

% Preamble
\documentclass[title={January 22, 2024 Notes}]{cs581notes}
\usepackage{animate}

% Document
\begin{document}

%<*1-22-2024>

\section{A* Evaluation Function}\label{sec:a*-evaluation-function}
\[ f(n) = g(\mbox{State}_{n}) + h(\mbox{State}_{n}) \]
where:
\begin{itemize}
	\item $g(n)$ -- initial node
\end{itemize}

\section{Admissible Heuristic: Proof}\label{sec:admissible-heuristic:-proof}
An admissible heuristics $h()$ is guaranteed to give you the optimal solution.
Why?
Proof by contradiction:
\begin{itemize}
	\item Say: the algorithm returned a suboptimal path $(C > C*)$
	\item So: there exists a node $n$ on $C*$ not expanded on $C$:
\end{itemize}

If so:
\begin{equation*}
\begin{aligned}
	f(n) &> C*&\\
	f(n) &= g(n) + h(n) & \mbox{(by definition)}\\
	f(n) &= g*(n) + h(n) & \mbox{(because } n \mbox{ is on } C*)\\
	f(n) &\leq g*(n) + h*(n) & \mbox{(if } h(n) \mbox{ admissible: } h(n) \leq h*(n)\\
\end{aligned}
\end{equation*}

\section{What Made A* Work Well?}\label{sec:what-made-a*-work-well?}
\begin{itemize}
	\item Straight-line heuristics is consistent: its estimate is getting better and better as we get closer to the goal
	\item Every consistent heuristics is admissible heuristics, but not the other way around
\end{itemize}

But that would mean that: \[ f(n) \leq C* \]

\section{A*: Search Contours}\label{sec:a*:-search-contours}
How does A* ``direct'' the search progress?

\section{Dominating Heuristics}\label{sec:dominating-heuristics}
We can have more than one available heuristics.
For example $h_{1}(n)$ and $h_{2}(n)$.
$h_{2}(n)$ dominates $h_{1}(n)$ iff\footnote{if and only if} $h_{2}(n) > h_{1}(n)$ for every $n$.

If you have multiple admissible heuristics where none dominates the other:
\[ \mbox{Let } h(n) = \max\left( h_{1}(n), h_{2}(n), \dots, h_{m}(n) \right) \]

\section{Domination $\rightarrow$ Efficiency: Why?}\label{sec:domination-rightarrow-efficiency:-why?}
With \[ f(n) < C* \mbox{ and } f(n) = g(\mbox{State}_{n}) + h(\mbox{State}_{n}),\]
we get

\section{Domination $\rightarrow$ Efficiency: But?}\label{sec:domination-rightarrow-efficiency:-but?}
If $h_{2}(n)$ dominates $h_{1}(n)$, should you always use $h_{2}(n)$?
Generally yes, but $h_{2}(n)$ vs $h_{1}(n)$ heuristic \emph{computation time} may be a deciding factor here.

\section{Heuristic and Search Performance}\label{sec:heuristic-and-search-performance}
\begin{itemize}
	\item Consider an 8-puzzle game and two admissible heuristics:
	\begin{itemize}
		\item $h_{1}(n)$ -- number of misplaced tiles (not counting blank)
		\item $h_{2}(n)$ -- Manhattan distance
	\end{itemize}
\end{itemize}

\section{$h()$ Quality: Effective Branching}\label{sec:$h()$-quality:-effective-branching}

\section{Can We Make A* Even Faster? (Sometimes at a cost!)}\label{sec:can-we-make-a*-even-faster?-(sometimes-at-a-cost!)}

\section{Weighted A* Evaluation Function}\label{sec:weighted-a*-evaluation-function}
\[ f(n) = g(\mbox{State}_{n}) + W * h(\mbox{State}_{n}) \]
where:
\begin{itemize}
	\item $g(n)$ -- initial node to node $n$ path cost
	\item $h(n)$ -- estimated cost of the best path that continues from node $n$ to a goal node
	\item $W > 1$
\end{itemize}

Here, weight $W$ makes $h(\mbox{State}_{n})$ (perhaps only ``sometimes'') inadmissible.
It becomes potentially more accurate = less expansions!

\animategraphics[autoplay,loop,width=\textwidth]
{10}%Frame rate
{file}%File without extension
{}
{}

%</1-22-2024>

\end{document}