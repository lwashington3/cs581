%! Author = Len Washington III
%! Date = 1/31/2024
%! compiler = pdflatex

% Preamble
\documentclass[title={January 31, 2024 Notes}]{cs581notes}

% Document
\begin{document}

%<*1-31-2024>

\newcommand{\representation}[2]{
	\begin{table}[H]
		\label{tab:representation-#1}
		\centering
		\begin{tabular}{|c|c|c|c|c|c|c|c|c|c|}
			\hline
			\multicolumn{10}{|c|}{#1}\\
			\hline
			#2\\
			\hline
		\end{tabular}
	\end{table}
}

\section{Local Beam Search}\label{sec:local-beam-search}
The local beam search algorithm:
\begin{itemize}
	\item keeps track of $k$ states rather than just one
	\item begins with $k$ randomly generated states
	\item at each step, all the successors of $k$ states
\end{itemize}

\subsection{The Idea}\label{subsec:the-idea}
In a local beam search useful information is passed among the $k$ parallel search threads.

For example, if one state generates several good successors and the other $k-1$ states all generate bad successors, then the effect is that

\section{Tabu Search}\label{sec:tabu-search}
\subsection{Key Features}\label{subsec:key-features}
\begin{itemize}
	\item Always move to the best \emph{available} neighborhood solution, even if it is worse than the current solution
	\item Maintain a list of solution points that must be avoided (not allowed) or a list of move features that are not allowed:
	\begin{itemize}
		\item This is the Tabu List.
	\end{itemize}
	\item Update the Tabu List based on some memory structure (short-term memory):
	\begin{itemize}
		\item Remove Tabu moves after some time period has elapsed (Tenure).
	\end{itemize}
	\item Allow for exceptions
\end{itemize}

\subsection{Memory Structures}\label{subsec:memory-structures}
The memory structures used in Tabu Search can be divided into three categories:
\begin{description}
	\item[Short-term] The list of solutions recently considered.
	If a potential solution appears on this list, it cannot be revisited until it reaches an expiration point (Tenure).
	\item[Intermediate-term] A list of rules intended to bias the search towards promising areas of the search space.
	\item[Long-term] Rules that promote diversity in the search process
	(i.e. regarding resets when the search becomes stuck in a plateau or a suboptimal dead-end).
\end{description}

\subsection{Aspiration Criteria}\label{subsec:aspiration-criteria}
A criteria which allows a Tabu move to be accepted under certain conditions.

Most common aspiration criterion: If the move finds a new best solution, then accept the move even if the move is Tabu.

\subsection{Tenure}\label{subsec:tenure}
The Tabu Tenure is the number of iterations that a move stays in the Tabu List
\begin{description}
	\item[too small] risk of cycling
	\item[too large] may restrict the search too much
\end{description}

\subsection{Intensification}\label{subsec:intensification}
Search parameters can be locally modified in order to perform intensification and/or diversification

\definition{Intensification}{usually applied when no configurations with a quality comparable to that of stored elite configuration(s), have been found in the }

\subsection{Stopping Criteria}\label{subsec:stopping-criteria}
Potential Stopping Criteria:
\begin{itemize}
	\item Number of iterations
	\item Number of iterations without improvement
	\item
	\item
\end{itemize}

\chapter{Evolutionary Algorithms}\label{ch:evolutionary-algorithms}
\section{Evolutionary Algorithms [Wikipedia]}\label{sec:evolutionary-algorithms-[wikipedia]}
An evolutionary algorithm (EA) is a subset of evolutionary computation, a generic population-based metaheuristic optimization algorithm.\\

An EA uses mechanisms inspired by biological evolution, such as reproduction, mutation, recombination, and selection.
Candidate solutions to the optimization problem play the role of individuals in a population, and the fitness function determines the quality of the

\section{Chromosome}\label{sec:chromosome}
\definition{Chromosome}{a package of DNA with part of all of the genetic materials of an organism}

\section{Artificial Chromosome}\label{sec:artificial-chromosome}
In Evolutionary Algorithms, an artificial chromosome is a genetic representation of the task to be solved.\\

Typically:
\begin{center} 1 individual = 1 chromosome = 1 solution \end{center}
Also called a genotype.

	%</1-31-2024>

\subsection{Representation}\label{subsec:chromosome-representation}
Individuals / chromosomes can be represented as a string of values.
Typically:
\begin{itemize}
	\item Binary
	\representation{Binary}{0 & 1 & 0 & 0 & 1 & 1 & 1 & 0 & 1 & 1}
	\item Integer
	\representation{Integer}{2 & 1 & 11 & 2 & 3 & 78 &  &  &  & }
	\item Floating-point
	\representation{Float}{ &  &  &  &  &  &  &  &  & }
\end{itemize}

\section{Well-Suited Chromosome: Features}\label{sec:well-suited-chromosome:-features}
\begin{itemize}
	\item It must allow the accessibility of all admissible points in the search space.
	\item Design the chromosome in such a way that it covers only the search space and no additional areas so that there is no redundancy or only as little
\end{itemize}

\section{Genotype vs Phenotype}\label{sec:genotype-vs-phenotype}
\begin{minipage}[t]{0.5\textwidth}
	\definition{Genotype}{Organism's full hereditary information, even if not expressed. Directly inherited from parents}
\end{minipage}\begin{minipage}[t]{0.5\textwidth}
\end{minipage}

\section{Genes vs Alleles}\label{sec:genes-vs-alleles}
\begin{minipage}[t]{0.5\textwidth}
	\definition{Genotype}{Organism's full hereditary information, even if not expressed. Directly inherited from parents}
\end{minipage}\begin{minipage}[t]{0.5\textwidth}
\end{minipage}

\section{Population}\label{sec:population}
The set of solutions (individuals / chromosomes / genotypes) is called a population.

\section{Individual Selection Mechanisms}\label{sec:individual-selection-mechanisms}

\section{Crossover Mechanisms}\label{sec:crossover-mechanisms}

\section{Potential Mutation}\label{sec:potential-mutation}
\subsection{Mutation / Probability of Mutation}\label{subsec:mutation-/-probability-of-mutation}
\begin{itemize}
	\item Each component (bit, etc.) of every individual / chromosome is modified with \emph{mutation probability $m_{p}$}
\end{itemize}

\begin{algorithm}[H]
	\caption{Genetic Algorithm Pseudocode}\label{alg:genetric-algorithm}
	\begin{algorithmic}[1]
	\Function{Genetric-Algorithm}{$population,\ fitness$} \Returns an individual
		\Repeat
			\State $weights\gets\Call{Weighted-By}{population, fitness}$
			\State $population2\gets$ empty list
			\For{$i=1$ to $\Call{Size}{population}$}
				\State $parent1,\ parent2\gets\Call{Weighted-Random-Choices}{population, weights, 2}$
				\State $child\gets\Call{Reproduce}{parent1,\ parent2}$
				\If{small random probability}
					\State $child\gets\Call{Mutate}{child}$
				\EndIf
				\State add $child$ to $population2$
			\EndFor
			\State $population\gets population2$
		\Until{some individual is fit enough, or enough time has elapsed}
		\State \Return the best individual in $population$, according to $fitness$
	\EndFunction
	\end{algorithmic}
\end{algorithm}

\section{Genetic Algorithms}\label{sec:genetic-algorithms}
\subsection{Design Issues}\label{subsec:design-issues}
\begin{itemize}
	\item Choosing basic implementation issues:
	\begin{itemize}
		\item representation
		\item
		\item
		\item
	\end{itemize}
	\item Termination criteria
	\item Performance, scalability
	\item Solution is only as good as the evaluation function (often the hardest part)
\end{itemize}

\subsection{Benefits}\label{subsec:genetic-algorithms-benefits}
\begin{itemize}
	\item Easy to understand and implement
	\item Modular, separate from application
	\item Supports multi-objective optimization
	\item Good for ``noisy'' environments
	\item Always has an answer
	\begin{itemize}
		\item Answers get better with time
	\end{itemize}
	\item Inherently parallel $\rightarrow$
	\item Variety of ways to improve performance as knowledge about the problem domain is gained
	\item Can exploit historical / alternative solutions
	\item Can be easily
\end{itemize}

\subsection{When to Use}\label{subsec:when-to-use}
\begin{itemize}
	\item Other solutions too slow or overly complicated (intractable mathematically)
	\item As an exploratory tool to examine new approaches
\end{itemize}

\subsection{Selected GA Applications}\label{subsec:selected-ga-applications}


\end{document}