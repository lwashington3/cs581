%! Author = Len Washington III
%! Date = 2/12/2024
%! compiler = pdflatex

% Preamble
\documentclass[title={February 12, 2024 Notes}]{cs581notes}

% Document
\begin{document}

%<*2-12-2024>
\maketitle

\chapter{Genetic Algorithm}\label{ch:genetic-algorithm}
\section{Fitness Function}\label{sec:fitness-function}
\begin{itemize}
	\item Has to be clearly
\end{itemize}

\section{Evolutionary Algorithms: Speciation}\label{sec:evolutionary-algorithms:-speciation}
\begin{itemize}
	\item \definition{Nature}{speciation occurs when two similar reproducing beings evolve to become too dissimilar to share genetic information effectively or correctly.}
	\begin{itemize}
		\item they are incapable of mating to produce offspring.
		\begin{itemize}
			\item Example: a horse and a donkey mating to produce a mule.
			However, in this case, the Mule is usually infertile, and so the genetic isolation of the two parent species is maintained.
		\end{itemize}
	\end{itemize}
	\item \definition{Implementation}{speciation$\rightarrow$}
\end{itemize}

\chapter{Genetic Programming}\label{ch:genetic-programming}
\begin{itemize}
	\item Genetic programming (GP) is an automated method for creating a working computer program from a high-level problem statement of a problem.
	\item Genetic programming starts from a high-level statement of ``what needs to be done'' and
\end{itemize}

\begin{description}
	\item[Goal:] find a program that generates the correct solution
	\begin{itemize}
		\item based on input -- correct output data
	\end{itemize}
	\item[Genotype:]
\end{description}

\section{Preparation}\label{sec:preparation}
\begin{itemize}
	\item Determine the set of terminals.
	\item Select the set of primitive functions.
	\item Define the fitness function.
	\item Decide on the parameters for controlling the run.
	\item Choose the method for designating a result of the run.
\end{itemize}

\section{Problem Description}\label{sec:problem-description}
\begin{table}[H]
    \centering
	\caption{}
	\label{tab:}
	\begin{tabular}{|*{2}{p{0.5\textwidth}|}}
		\hline
		Objective & Find a computer program with one input $X$ for which the output $Y$\\
		\hline
	\end{tabular}
\end{table}

\section{Mutation}\label{sec:mutation}
Randomly choose :

\chapter{Other (Key) Evolutionary Approaches}\label{ch:other-(key)-evolutionary-approaches}
\section{Evolutionary Programming}\label{sec:evolutionary-programming}
\begin{itemize}
	\item Similar to genetic programming, but the solution is a set parameters for a predefined fixed computer program, not a generated computer program
	\item Solution fitness is determined by how well the fixed
\end{itemize}

\section{Fuzzy Logic}\label{sec:fuzzy-logic}
Imagine load application processing rules

\begin{center}
	If credit score good then risk is low\\
	If credit score bad then risk is high\\
	If credit score medium then risk is average\\
\end{center}

What does it mean: low, high, good bad?

\section{Genetic Fuzzy Systems}\label{sec:genetic-fuzzy-systems}

\section{Swarm Intelligence/Optimization}\label{sec:swarm-optimization}
Swarm Intelligence (SI) is the collective behavior of decentralized, self-organized systems, natural or artificial.
The concept is employed in work on artificial intelligence.

\section{Emergence}\label{sec:emergence}
In philosophy, systems theory, science, and art, emergence occurs when

\section{Ant Colony Optimization}\label{sec:ant-colony-optimization}
\subsection{Stigmergy}\label{subsec:stigmergy}
Stigmergy (coined by French biologist Pierre-Paul Grasse) = interation through the environment

\subsection{Pheromone Trail}\label{subsec:pheromone-trail}

\subsection{TSP with Ant Colony Optimization}\label{subsec:tsp-with-ant-colony-optimization}
\begin{enumerate}
    \item Initialization (ants, pheromone trails)
	\item Randomly place ants at nodes
	\item Build tours / paths
	\item Deposit pheromone, update trail, solution
	\item Repeat or exit
\end{enumerate}

\subsection{Select Next Node For Each Ant}\label{subsec:select-next-node-for-each-ant}
Each ant will pick its next destination:
\begin{itemize}
	\item unvisited node
	\item choice will be based on:
	\begin{itemize}
		\item pheromone intensities $d_{ij}$ on all available paths
		\item heuristic value $h_{ij}$ for all available paths (a distance between nodes)
	\end{itemize}
\end{itemize}

\[ P(move\ i\rightarrow j) = \frac{d_{ij}^{\alpha}\times\frac{1}{h_{ij}}^{\beta}}{\sum_{k=1}^{N\ possible\ destinations} d_{ik}^{\alpha}\times\frac{1}{h_{ik}}^{\beta}} \]
where
\begin{description}
	\item[$\alpha$ =]pheromone weight ``intensity''
	\item[$\beta$ =]
\end{description}

\subsection{Update Pheromone Trail}\label{subsec:update-pheromone-trail}
Update pheromone trail values for each edge:
\begin{itemize}
	\item decrease due to pheromone evaporation (evaporation rate, for example 0.5)
	\[ *= 0.5 \]
	\item
\end{itemize}

%</2-12-2024>

\end{document}